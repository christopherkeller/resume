%!TEX root=index.tex

\IfFileExists{contact}{%
    \input{contact}
}

\hrule\
\vspace{-0.4em}
\subsection*{Core}
\begin{indentsection}{\parindent}
    \hyphenpenalty=1000
    \begin{description*}
        \item[GitHub:] https://github.com/christopherkeller, https://github.com/datanexus
        \item[Languages:] C, Python, Go, Unix Shell, \textsc{SQL}
        \item[Databases:] My\textsc{SQL}, Postgre\textsc{SQL}, NoSQL Cassandra
        \item[Cloud:] AWS, Azure, OpenStack, KVM, Openshift
        \item[DevOps:] Git, Jenkins, Docker, Ansible
        \item[Concepts:] scalability \& availability, data, microservices, technical strategy \& leadership, configuration management
	\end{description*}
\end{indentsection}
\hrule\
\vspace{-0.4em}

\subsection*{Patents}

\begin{itemize}
    \parskip=0.1em
    \item
    \headerrow
        {\textbf{2018 Orchestration of cloud based data collection, integration, and routing} {\emph{Co-Author}}}
        {\textbf{DataNexus}}
  
\end{itemize}

\hrule\
\vspace{-0.4em}

\subsection*{Experience}
\begin{itemize}
    \parskip=0.1em
      \item\
          \headerrow {\textbf{DataNexus, Inc}} {\textbf{Colorado Springs, CO \& Greenville, SC}}
          \headerrow {\emph{Chief Technology Officer}} {\emph{Since 2016}}
          \begin{itemize*}
            \item Responsible for product architecture and engineering of distributed streaming data platform
            \begin{itemize}
              \item Automated installation \& configuration of core components, networking, and API web services: Go, Python, Ansible, PostgreSQL, Cassandra, Zookeeper, Kafka, Solr, Elasticsearch, Hadoop
              \item Commercial overlays offering TLS encryption and key/certificate lifecycle management 
              \item Integration and intelligent orchestration with public and on-premise cloud offerings and components: AWS, Azure, OpenStack, Openshift, KVM, S3, Azure IOT Hub
              \item Develop in-stream MDM approach utilizing filtering and masking: multi-process Python, KSQL
            \end{itemize}
            \item Architecture for DataTrust blockchain to address consumer data privacy laws (GDPR): Hyperledger
            \item Develop customer focused classes: mentorship, open source, CM, CI/CD, devops, and embracing a culture of change
            \item Success story: replaced legacy systems with a robust hybrid cloud and MDM data platform utilizing Agile methods
            \item Success story: Solved performance and scalability issues by installing components from DataNexus platform
          \end{itemize*}
         
	\item
    \headerrow
    {\textbf{ClimateMonkeys}}
    {\textbf{Damascus, MD}}

    \headerrow
		{\emph{Advisor}}
		{\emph{Since 2015}}
	\begin{itemize*}
		\item Technical and business advisor to founding team; primarily focused on building a climate data platform focusing on scalability
	\end{itemize*}
  
    \item\
        \headerrow {\textbf{CSC}} {\textbf{Falls Church, VA}}
        \headerrow {\emph{Senior Principal Architect, Office of the CTO Global Infrastructure}} {\emph{2016}}
        \begin{itemize*}
            \item Devise and implement technical strategy for offerings
                \begin{itemize*}
                  \item Focus on infrastructure, scaleable configuration management practices, and development automation: Enterprise GitHub, Ansible Tower, Docker
                \end{itemize*}
            \item Continue to lead R\&D on emerging technologies
        \end{itemize*}
        \headerrow {\emph{Senior Principal Architect, Office of the CTO Emerging Business Offerings}} {\emph{2015}}
        \begin{itemize*}
            \item Serve in backup role for the CTO for Emerging Business focusing on: training \& enablement, big data, cyber security, mobility, social, IoT, cloud, next generation infrastructure
            \begin{itemize*}
              \item Build out data center for CoE for Emerging Technologies in Austin, Texas
              \item Solve and deliver PCI, FISMA, HIPAA compliance certifications for Cloudera \& HortonWorks
            \end{itemize*}
            \item Evaluate and recommend acquisition targets based on alignment with global strategy
            \item Responsible for integration of partner and vendor technologies into CSC reference architectures and go-to-market strategies    
              \begin{itemize*}
                \item Dell, Dispersive Technologies, EMC, Intel, Lenovo, Cloudera, DataStax, Docker, HortonWorks, RedHat, Arista, Brocade, Cisco
              \end{itemize*}
        \end{itemize*}
        \headerrow {\emph{Principal Architect, Office of the CTO Big Data \& Analytics}} {\emph{2014}}
        \begin{itemize*}
            \item Primary architect for the Big Data Platform as a Service: Infochimps Hadoop, AWS, OpenStack
        \end{itemize*}
        \headerrow {\emph{Principal Solutions Architect, Big Data \& Analytics}} {\emph{2013 - 2014}}
        \begin{itemize*}
            \item Assist in the design and creation of a big data solution and delivery center: Hadoop, Infochimps
            \item Lead architect for ClimatEdge\texttrademark\ : Python, C, MySQL, R, Jenkins
            \item Lead architect on SOR data acquisition project for internal analytical platform
        \end{itemize*}
        \headerrow {\emph{Principal Systems Security Leader}} {\textbf{NASA Ames Research Center, CA}}
        \headerrow {} {\emph{2007--2013}}
    \begin{itemize*}
        \item Mentor staff, lead implementation efforts, deliver strategy, and ensure on--time delivery of all projects while maintaining adherence to all Federal compliance mandates: FISMA, FDCC, NIST 800-53
        \item Architect and build a scalable event management system: Cassandra, Python, Jenkins, Rails
        \item Design virtualized infrastructure, ReSTful APIs, and web services to deliver security relevant data in real--time: Xen, Puppet, Python, PHP
	\end{itemize*}

  \item
  \headerrow
      {\textbf{Advanced Management Technology, Inc}}
      {\textbf{NASA Ames Research Center, CA}}
	\headerrow
		{\emph{Senior Systems Architect, Information Security}}
		{\emph{2006--2007}}
	\begin{itemize*}
        \item Develop a perimeter security architecture for protecting NASA Top 500 supercomputers: Linux, C, Shell
        \item Oversee ARC first certifications for NIST, FISMA, and HSPD--12 compliant security systems along with NASA first ArcSight SIEM installation
	\end{itemize*}

	\headerrow
    {\emph{Senior Systems Architect, Tools \& Infrastructure}}
    {\emph{2004--2005}}
    \begin{itemize*}
        \item Refactor tools group with a focus on practical applications of technology to NASA, efficiency and cost savings: MySQL, Sybase, Python, CMMI, OTRS
    \end{itemize*}

    \headerrow
    {\emph{Senior Software Engineer, Information Power Grid}}
    {\emph{2003--2004}}
    \begin{itemize*}
        \item Develop NASA’s Information Power Grid and Marshall Space Flight Center ISS science portal: Globus Toolkit
    \end{itemize*}

	\item
    \headerrow
    {\textbf{Interclypse}}
    {\textbf{San Jose, CA}}

    \headerrow
		{\emph{Founder}}
		{\emph{2003--2004}}
	\begin{itemize*}
		\item Founded company to provide an aggressive open source migration strategy to U.S. DoD: Linux kernel tuning, Oracle RAC, Solaris
	\end{itemize*}

	\item
	\headerrow
		{\textbf{BeamReach Networks, Inc}}
		{\textbf{Sunnyvale, CA}}
	\\
	\headerrow
		{\emph{Senior Systems \& Software Engineer}}
		{\emph{2001--2003}}
	\begin{itemize*}
		\item After significant staff reduction, performed all aspects of CIO role, including security threat assessments
		\item Architect and program a multi-tiered testing framework for Integration and Test effort: Postgres, Oracle 8i/9i, C, PHP, Java
	\end{itemize*}

	\item
	\headerrow
		{\textbf{ePropose, Inc}}
		{\textbf{San Francisco, CA}}
	\\
	\headerrow
		{\emph{Senior Software Engineer}}
		{\emph{2000--2001}}
	\begin{itemize*}
		\item Lead engineer in Release Management Infrastructure group; automated generation of API's, continuous delivery, and automated platform builds
		\item Developer in Professional Services; analyzed client requirements and developed custom changes as needed: Java
	\end{itemize*}

	\item
	\headerrow
		{\textbf{Science Applications International Corporation}}
		{\textbf{Annapolis, MD}}
	\\
	\headerrow
		{\emph{Senior Software Engineer}}
		{\emph{1995--2000}}
	\begin{itemize*}
		\item Task lead, and data \& algorithm architect on world's largest classified Oracle installation: Solaris, NetApp, Java, Perl, HTML, Bourne/C shell, \CPP, Perl, JavaScript, HTML, Cisco, XML, XSL-T, CSS
		\item Remote development lead for NIMA's imagery system: Perl, C, \CPP, X/Motif, Sybase
	\end{itemize*}

	\item\
	\headerrow
		{\textbf{University of Maryland, Department of Astronomy}}
		{\textbf{Goddard Space Flight Center, MD}}
	\\
	\headerrow
		{\emph{Research Assistant}}
		{\emph{1991--1995}}
	\begin{itemize*}
		\item Assist University of Maryland professor in geophysical and remote sensing research by programming an autonomous ETL system: HPUX, C, Unix Shell, IDL
	\end{itemize*}
\end{itemize}

\hrule
\vspace{-0.4em}

\subsection*{Awards}

\begin{itemize}
	\parskip=0.1em
	\item
	\headerrow
		{\textbf{2017 Global Innovation} for {\emph{Cloud Innovation}} \& {\emph{Open Source Way}}}
		{\textbf{RedHat}}
  \item
	\headerrow
		{\textbf{2014 Papers Winner} for {\emph{A Big Data Approach to ClimatEdge\texttrademark}}}
		{\textbf{CSC}}

	\item
	\headerrow
		{\textbf{2010 Group Honor Award} for {\emph{Security Architecture and Development Innovation}}}
		{\textbf{NASA Ames Research Center}}

	\item
	\headerrow
		{\textbf{2008 Group Achievement Award} for {\emph{IT Security Practices}} }
		{\textbf{NASA Ames Research Center}}

	\item
	\headerrow
		{\textbf{2007 Certificate of Appreciation} for {\emph{Outstanding Performance}}}
		{\textbf{NASA Ames Research Center}}

	\item
	\headerrow
		{\textbf{2007 Commendation} for {\emph{Data Warehouse Design, Development, Delivery}} }
		{\textbf{NASA Shuttle Orbiter Team}}

	\item
	\headerrow
		{\textbf{2005 Employee Award} for {\emph{Outstanding Performance}} }
		{\textbf{Advanced Management Technology, Inc}}

	\item
	\headerrow
		{\textbf{2005 Certificate of Excellence} for {\emph{Project Columbia}}}
		{\textbf{NASA Advanced Supercomputing Division}}

	\item
	\headerrow
		{\textbf{2005 Group Achievement Award} for {\emph{Project Columbia}}}
		{\textbf{NASA Ames Research Center}}

	\item
	\headerrow
		{\textbf{2004 Certificate of Excellence} for {\emph{IT Security}}}
		{\textbf{NASA Ames Research Center}}

	\item
	\headerrow
		{\textbf{1997 \& 1998 Various} for {\emph{Redacted}}}
		{\textbf{US Department of Defense}}
    
\end{itemize}

\hrule
\vspace{-0.4em}

\subsection*{Presentations \& Publications}

\begin{itemize}
    \parskip=0.1em
    \item
    \headerrow
        {\textbf{2018 DataTrust Identity Ledger} {\emph{Author}} }
        {\textbf{DataNexus}}
 
    \item
    \headerrow
        {\textbf{2013 A Big Data Approach to ClimatEdge\texttrademark} {\emph{Author}}}
        {\textbf{CSC NSL Institute Thesis}}

    \item
	\headerrow
		{\textbf{2013 COTS to Cassandra} {\emph{Author}}}
		{\textbf{Software Developers Journal} {\emph{Vol 2, No. 1 2013}}}
		
    \item
	\headerrow
		{\textbf{2012 Cassandra Users Group} presenting {\emph{Real World Cassandra at NASA}}}
		{\textbf{Vienna, VA \& San Mateo, CA}}

	\item
	\headerrow
		{\textbf{2012 Investor Big Data Day} {\emph{Panelist}}}
		{\textbf{New York, NY}}
	
	\item
    \headerrow
		{\textbf{2010 CENIC Conference Speaker} presenting {\emph{Representational State Transfer (REST) in Practice}}}
		{\textbf{Monterey, CA}}
    
	\item
	\headerrow
		{\textbf{2006 The Tension Between Strong Perimeter Control and Usability} {\emph{Co-author}}}
		{\textbf{NASA Ames Research Center, CA}}

	\item
	\headerrow
		{\textbf{1996 Java By Example} {\emph{Contributor}}}
		{\textbf{Sams Publishing}}

\end{itemize}

% \IfFileExists{clearances}{%
%     \input{clearances}
% }

\hrule\
\vspace{-0.4em}
\subsection*{Education}

\begin{itemize}
	\parskip=0.1em

	\item
	\headerrow
		{\textbf{CSC NPS Solution Leadership Institute}}
		{\textbf{Falls Church, VA}}
	\headerrow
		{\emph{Solutions Architecture}}
		{\emph{2012 -- 2013}}
	
	\item
	\headerrow
		{\textbf{University of Maryland \& University of Maryland, University College}}
		{\textbf{College Park, MD}}
	\headerrow
		{\emph{Department of Computer Science, B.S. Computer Science}}
		{\emph{1990 -- 1996}}
	
\end{itemize}
